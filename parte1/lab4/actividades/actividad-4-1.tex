\subsubsection{Actividad 1_lab4}

%*********************
\begin{frame}
\pgfdeclareimage[width=\paperwidth,height=\paperheight]{bg}{imagenes/fondo_seccion}
\setbeamertemplate{background}{\pgfuseimage{bg}}

\definecolor{greenU}{RGB}{212,202,72}
\setbeamercolor{block body}{fg=Black,bg=greenU}
\begin{block}{}
\centering
\vspace{8mm}
\Large{Actividades}
\vspace{8mm}
\end{block}
\end{frame}
%********************

\begin{frame}

\pgfdeclareimage[width=\paperwidth,height=\paperheight]{bg}{imagenes/fondo3}
\setbeamertemplate{background}{\pgfuseimage{bg}}

\frametitle{\underline{\textbf{¿Comó Observar un diagrama de constelación?}}}

\begin{flushleft}
Para visualizar la constelación en Gnu-Radio podemos utilizar los siguientes bloques:
\end{flushleft}
  
\begin{itemize}
\item {
QT GUI Constellation Sink: Permite mostrar la cosntelación I-Q de multiples señales, su   receptor admite el trazado de transmisión de datos o mensajes complejos, el puerto de entrada para el mensaje se denomina "in". }
\end{itemize}
\begin{itemize}
\item {
WX GUI Contellation Sink : es la representación gráfica de las señales complejas, es  útil  para  el  estudio  y  visualización  de  modulaciones digitales  como:  M-PSK, QAM, FSK, ASK}
\end{itemize}
\end{frame}
%-----------------------------------

\begin{frame}
\begin{flushleft}
Los diagramas de constelación se pueden utilizar para reconocer la interferencia y distorsión existente en una señal, por otra parte también se puede representar la relación de amplitud y fase de una portadora modulada digitalmente. 
\item Responda: ¿Se puede ver el diagrama de constelaciones de la modulación ASK?
 ¿ y que se debe observar?
\end{flushleft}
\end{frame}

%-----------------------------------

