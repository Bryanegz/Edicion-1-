\subsubsection{Actividad_1_lab_5}

%*********************
\begin{frame}
\pgfdeclareimage[width=\paperwidth,height=\paperheight]{bg}{imagenes/fondo_seccion}
\setbeamertemplate{background}{\pgfuseimage{bg}}

\definecolor{greenU}{RGB}{212,202,72}
\setbeamercolor{block body}{fg=Black,bg=greenU}
\begin{block}{}
\centering
\vspace{8mm}
\Large{Actividades}
\vspace{8mm}
\end{block}
\end{frame}
%********************

%--------------------------------------------------------------------------------------------
\begin{frame}{Flujo de datos digitales BPSK}
\pgfdeclareimage[width=\paperwidth,height=\paperheight]{bg}{imagenes/fondo3}
\setbeamertemplate{background}{\pgfuseimage{bg}}

\frametitle{Flujo de datos digitales BPSK}
\framesubtitle{Actividad}
¿Qué efectos produce el parámetro $\alpha$ del filtro de raíz de coseno realzado sobre la señal y el ancho de banda?\\
\begin{itemize}
\item Grafique en un bloque de WX GUI FFT Sink la salida del segundo bloque de Root Raised Cosine Filter. Recuerde antes convertir los datos de flotantes a tipo complejo.     
\item Realice cambios al parámetro $\alpha$ del bloque Root Raised Cosine Filter, y observe los cambios en el FFT Plot y en el Scope Plot.
\end{itemize}

\end{frame}
%----------------------------------------------------------------------------
