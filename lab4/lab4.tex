\section{Lab4: Modulación ASK en GRC}

%*********************
\begin{frame}{}

\pgfdeclareimage[width=\paperwidth,height=\paperheight]{bg}{imagenes/fondocap2}
\setbeamertemplate{background}{\pgfuseimage{bg}}

\bfseries{\textrm{\LARGE Lab4\\ \Large Modulación ASK en GRC}}
\raggedright
\end{frame}
%*********************

\begin{frame}{Modulación ASK en GRC}

\pgfdeclareimage[width=\paperwidth,height=\paperheight]{bg}{imagenes/fondo3}
\setbeamertemplate{background}{\pgfuseimage{bg}}

\begin{itemize}
  \item {
La Modulación por desplazamiento de amplitud (ASK) es un
esquema de modulación digital en el que la amplitud de la
onda portadora se cambia con respecto a la señal de
información, manteniendo la fase y la frecuencia de constantes.
Para el presente laboratorio, se utilizó BASK, el cual tiene el
mismo comportamiento, pero utilizando bits. Donde su
comportamiento es descrito por la siguiente ecuación:

$$S(t)=Am(t)cos(wt)$$

  }
  \item {
El diagrama de boques en GRC consiste en la multiplicación de
una señal de información con una portadora, que corresponde
a una señal coseno de diez veces la frecuencia de la
información. La frecuencia de muestreo debe ser mayor que el
doble de la frecuencia máxima de la señal de datos.
  }
  \end{itemize}

\end{frame}

\begin{frame}{Modulacion ASK en GRC}
\begin{figure}[H]
\centering
\includegraphics[width=\textwidth]{lab4/pdf/141.pdf}
\end{figure}
\end{frame}

\begin{frame}{Modulacion ASK en GRC}
\begin{figure}[H]
\centering
\includegraphics[width=\textwidth,height=0.7\paperheight]{lab4/pdf/142.pdf}
\end{figure}
\end{frame}


\begin{frame}{Modulación ASK en GRC}
  \begin{itemize}
  \item {
La Modulación por desplazamiento de frecuencia (FSK) es una
técnica de modulación digital, utilizada para la transmisión de
datos. Para BFSK que corresponde al mismo proceso, pero con
datos binarios, se utilizan dos frecuencias diferentes para
transmitir dos señales, es decir 0 y 1. como se puede ver a
continuación:
$$ S_{1}(t)=Acos(w_{1}t)$$
$$S_{2}(t)=Acos(w_{2}t)$$
  }
  \item {
El diagrama de boques en GRC se utilizaron dos señales
coseno como portadora de amplitud 1, y frecuencias de 1Khz y
10Khz. La señal 1 es representada por una onda cuadrada, y la
señal 0 es obtenida restándole 1 a la señal cuadrada.
Posteriormente se multiplican las señales 0 y 1 con las
portadoras y luego se suman obteniendo la modulación FSK.
  }
  \end{itemize}
\end{frame}


\begin{frame}{Modulación ASK en GRC}
\begin{figure}[H]
\centering
\includegraphics[width=\textwidth]{lab4/pdf/143.pdf}
\end{figure}
\end{frame}

\begin{frame}{Modulacion ASK en GRC}
\begin{figure}[H]
\centering
\includegraphics[width=\textwidth,height=0.7\paperheight]{lab4/pdf/144.pdf}
\end{figure}
\end{frame}

\begin{frame}{Modulación ASK en GRC}
  \begin{itemize}
  \item {
La Modulación por desplazamiento de fase (PSK) es un
esquema de modulación digital donde se varia la fase de la señal, manteniendo la frecuencia y la amplitud constante. Para la modulación PSK se usan dos señales con fases diferentes y frecuencias iguales, estas se multiplican con la señal 0 o 1 de los datos como se muestra a continuación:

$$S_{1}(t)=Acos(wt)$$
$$S_{2}(t)=Asin(wt)$$

  }
  \item {
El diagrama de boques en GRC se utiliza una señal coseno de
50000 hertz y la amplitud pico de 1, y otra onda sinusoidal de
igual frecuencias y amplitud. Similar a BFSK se representa la
señal 1 con la onda cuadrada de frecuencia 5000 hertz y el 0
por la resta de la constante.
  }
  \end{itemize}
\end{frame}

\begin{frame}{Modulación ASK en GRC}
\begin{figure}[H]
\centering
\includegraphics[width=\paperwidth]{lab4/pdf/145.pdf}
\end{figure}
\end{frame}

\begin{frame}{Modulación ASK en GRC}
\begin{figure}[H]
\centering
\includegraphics[width=\textwidth,height=0.7\paperheight]{lab4/pdf/146.pdf}
\end{figure}
\end{frame}