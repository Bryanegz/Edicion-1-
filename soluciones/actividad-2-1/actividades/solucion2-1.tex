%\usepackage{amssymb}
%\usepackage{amsmath}
\subsubsection{Actividad 1 lab 2}
%*********************
\begin{frame}{}

\pgfdeclareimage[width=\paperwidth,height=\paperheight]{bg}{imagenes/fondo_seccion}
\setbeamertemplate{background}{\pgfuseimage{bg}}

\definecolor{greenU}{RGB}{212,202,72}
\setbeamercolor{block body}{fg=Black,bg=greenU}
\begin{block}{}
	\centering
	\vspace{1mm}
	\large{\textit{solucion lab 2 actividad 1}}
	\vspace{1mm}
\end{block}
\end{frame}
%*********************


\begin{frame}{Respuesta Actvidad 1 lab 2}
\begin{figure}[H]
\begin{flushleft}
\textbf{El ruido Gaussiano} se encuentra asociado con la radiación electromagnética. Ya que no podemos tener comunicación eléctrica sin electrones es imposible evitar el ruido, el ruido Gaussiano muestra una densidad de probabilidad que responde a una distribución normal(o distribución de Gauss).
\end{flushleft}
$f(x)={ \frac { 1 }{ \sigma \sqrt { 2\Pi  }  } e }^{ -\frac { 1 }{ 2 } { \left( \frac { x-u }{ \sigma  }  \right)  }^{ 2 } }$
\begin{flushleft}
\textbf{la distribución uniforme }continua es una familia de distribuciones de probabilidad para variables aleatorias continuas, tales que para cadamiembro de la familia, todos los intervalos de igual longitud en la distribución en su rango son igualmente probables. El dominio está definido por dos parámetros, a y b, que son sus valores mínimo y máximo. La distribución es a menudo escrita en forma abreviada como
\end{flushleft}
$f(x)=\left\{ \begin{matrix} \frac { 1 }{ b-a } \quad si\quad x\in (a,b) \\ 0\qquad si\quad x\notin (a,b) \end{matrix} \right\} $

\end{figure}
\end{frame}
%--------------------------------

\begin{frame}{Respuesta Actvidad 1 lab 2}
\begin{figure}[H]
\begin{flushleft}
\textbf{distribución de Laplace} es una densidad de probabilidad continua, llamada así en honor a Pierre-Simon Laplace. Es también conocida como distribución doble exponencial puesto que puede ser considerada como la relación las densidades de dos distribuciones exponenciales adyacentes. La distribución de Laplace resulta de la diferencia de dos variables exponenciales aleatorias, independientes eidénticamente distribuidas
\end{flushleft}
$f(x)=\left\{ \quad \quad \begin{matrix} { e }^{ \left( -\frac { \mu -x }{ b }  \right)  }\quad si\quad x\quad <\quad \mu  \\ { e }^{ \left( -\frac { x-\mu  }{ b }  \right)  }\quad si\quad x\quad \ge \quad \mu  \end{matrix} \right\} $
\begin{flushleft}
\textbf{El ruido impulsivo} es aquel ruido cuya intensidad aumenta bruscamente durante un impulso. La duración de este impulso es breve en comparación con  el tiempo que transcurre entre un impulso y otro. Incide fundamentalmente en la transmisión de los datos, se debebásicamente a fuertes inducciones 
consecuencias de conmutaciones electromagnéticas
\end{flushleft}
$\delta (x-c)=\left\{ \quad \quad \begin{matrix} \infty \quad si\quad x\quad =\quad c \\ 0\quad si\quad x\quad \neq \quad c \end{matrix} \right\} $

\end{figure}
\end{frame}
